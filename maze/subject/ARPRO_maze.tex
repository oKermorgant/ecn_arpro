\documentclass{ecnreport}
\setlength{\parindent}{0cm}

\stud{Option Robotique, Control \& Robotics master}
\topic{Advanced Robot Programming}

\def\maze{\okt{ecn::Maze}~}

\begin{document}

\inserttitle{Advanced Robot Programming  Labs \newline C++ Programming}

\insertsubtitle{Maze generation and solving}

\newcommand{\involves}[1]{
\item {\bf C++ skills:} #1
}

\newcommand{\aitip}[1]{
\item {\bf AI tips:} #1
}

\section{Content of this lab}

In this lab you will use and modify existing code in order to generate and solve mazes. As shown in \Fig{maze2}, the goal is to generate 
a maze of given dimensions (left picture) and to use a path planning algorithm to find the shortest path from the upper left to the lower right corners (right picture).

\begin{figure}[h]\centering
\includegraphics[width=.4\linewidth]{maze} \quad \quad\includegraphics[width=.4\linewidth]{maze_cell}
\caption{$51\times 75$ maze before (left) and after (right) path planning.}
\label{maze2}
\end{figure}

The lab was inspired by \link{https://www.youtube.com/watch?v=rop0W4QDOUI}{this Computerphile video}.\\

We use the classical mix of Git repository and CMake to download and compile the project:
\begin{center}
 \okt{mkdir build} $\rightarrow$ \okt{cd build} $\rightarrow$ \okt{cmake ..} $\rightarrow$ \okt{make}
\end{center}
The program can the be launched.


\section{Required work}

Four programs have to be created: 
\begin{enumerate}
 \item Maze generation
 \item Maze solving through A* with motions limited to 1 cell (already quite written)
 \item Maze solving through A* with motions using straight lines
 \item Maze solving through A* with motions using corridors
\end{enumerate}

As in many practical applications, you will start from some given tools (classes and algorithm) and use them inside your own code.\\
As told during the lectures, understanding and re-using existing code is as important as being able to write something from scratch.

\section{Maze generation}

The \link{https://en.wikipedia.org/w/index.php?title=Maze\_generation\_algorithm\&oldid=807735452\#C\_code_example}{Wikipedia page} on maze generation is quite complete
and also proposes C-code that generates a perfect maze of a given (odd) dimension. A perfect maze is a maze where there is one and only 
one path between any two cells.

Create a \okt{generator.cpp} file by copy/pasting the Wikipedia code and modify it so that:
\begin{itemize}
 \item It compiles as C++.
 \item The final maze is not displayed on the console but instead it is saved to an image file 
 \item The executable takes a third argument, that is the percentage of walls that are randomly erased in order to build a non-perfect maze.
\end{itemize}
A good size is typically a few hundred pixels height / width. To debug the code, 51 x 101 gives a very readable maze.

The \maze class (\Sec{mazeClass}) should be used to save the generated maze through its \okt{Maze::dig} method that removes a wall at a given (x,y) position.
It can also save a maze into an image file.


\section{Maze solving}

The given algorithm is described on \link{https://en.wikipedia.org/wiki/A*\_search\_algorithm\#Pseudocode}{Wikipedia}. It is basically a graph-search algorithm that
finds the shortest path and uses a heuristic function in order to get some clues about the direction to be favored.

In terms of implementation, the algorithm can deal with any \okt{Node} class that has the following methods:
\begin{itemize}
 \item \okt{vector<Node> Node::children()}: returns a \okt{vector} of \okt{Node} consisting of the children (or neighboors) of the considered
 element
 \item \okt{int distToParent()}: returns the distance to the node that generated this one
 \item \okt{bool operator==(const Node \&other)}: returns true if the passed argument is actually the same point
 \item \okt{double h(const Node \&goal) const}: returns the heuristic distance to the passed argument
 \item \okt{void show(bool closed, const Node \& parent)}: used for online display of the behavior
 \item \okt{void print(const Node \& parent)}: used for final display
\end{itemize}


While these functions highly depend on the application, in our case we consider a 2D maze so some of these functions are already implemented in, as seen in \Sec{ptClass}:
\begin{itemize}
 \item For the first exercice, only the \okt{children} method is to write.
 \item The second one adds the \okt{distToParent} method.
 \item The last one adds the \okt{show} and \okt{print} methods.
\end{itemize}



\subsection{A* with cell-based motions}

The first A* will use cell-based motions, where the algorithm can only jump 1 cell from the current one.

The file to modify is \okt{solve_cell.cpp}.
At the top of the file is the definition of a \okt{Position} class
that inherits from \okt{ecn::Point} in order not to reinvent the wheel (a point has two coordinates, it can compute the distance to
another point, etc.).

The only method to modify is \okt{Position::children} that should generates the neighboors of the current point.
The parent node is likely to be in those neighboors, but it will be removed by the algorithm. 

\subsection{A* with line-based motions}

Copy/paste the \okt{solve_cell.cpp} file to \okt{solve_line.cpp}.\\

Here the children should be generated so that a straight corridor is directly followed (ie children can only be corners, intersections or dead-ends).
A utility function \okt{bool is_corridor(int, int)} may be of good use.\\

The distance to the parent may not be always 1 anymore. As we know the distance when we look for the neighboor nodes, a good thing would be to store it 
at the generation by using a new Constructor with signature \okt{Position(int _x, int _y, int distance)}.\\

The existing \okt{ecn::Point} class is already able to display lines between two non-adjacent points (as long as they are on the same horizontal or
vertical line). The display should thus work directly.

\subsection{A* with corridor-based motions}

Copy/paste the \okt{solve_line.cpp} file to \okt{solve_corridor.cpp}.\\

Here the children should be generated so that any corridor is directly followed (ie children can only be intersections or dead-ends, but
not simple corners).
A utility function \okt{bool is_corridor(int, int)} may be of good use.\\

The distance to the parent may not be always 1 anymore. As we know the distance when we look for the neighboor nodes, a good thing would be to store it 
at the generation by using a new Constructor with signature \okt{Position(int _x, int _y, int distance)}. \\

The existing \okt{Point::show} and \okt{Point::print} methods are not suited anymore for this problem. Indeed, the path from a point to its parent
may not be a straight line. Actually it will be necessary to re-search for the parent using the same approach as to generate the children.\\

For this problem, remember that by construction the nodes can only be intersections or dead-ends. Still, the starting and goal positions may 
be in the middle of a corridor. It is thus necessary to check if a candidate position is the goal even if it is not the end of a corridor.

\section{Comparison}

Compare the approaches using various maze sizes and wall percentage.
If it is not the case, compile in \okt{Release} instead of \okt{Debug} and enjoy the speed improvement.\\

The expected behavior is that mazes with lots of walls (almost perfect mazes) should be solved must faster with the corridor, then line, then cell-based approaches.\\
With less and less walls, the line- and cell-based approaches should become faster as there are less and less corridors to follow.

\appendix

\section{Provided tools}

Do not modify the files associated with these classes.\\Their behavior is expected to be reproducible.

\subsection{The \maze class}\label{mazeClass}

This class interfaces with an image and allows easy reading / writing to the maze. 

\paragraph{Methods for maze creation}
\begin{itemize}
 \item \okt{Maze(std::string filename)}: loads the maze from the image file
 \item \okt{Maze(int height, int width)}: build a new maze of given dimensions, with only walls
 \item \okt{dig(int x, int y)}: write a free cell at the given position
 \item \okt{save()}: saves the maze to \okt{maze.png} and displays it
\end{itemize}

\paragraph{Methods for maze access}
\begin{itemize}
 \item \okt{int height()}, \okt{int width()}: maze dimensions
 \item \okt{int isFree(int x, int y)}: returns true for a free cell, or false for a wall or invalid (out-of-grid) coordinates
\end{itemize}

\paragraph{Methods for maze display}
\begin{itemize}
 \item \okt{write(int x, int y, int r, int g, int b, bool show = true)}\\
    will display the (x,y) with the (r,g,b) color and actually shows if asked
 \item \okt{passThrough(int x, int y)}: write the final path, color will go automatically from blue to red.
 Ordering is thus important when calling this function
 \item \okt{saveSolution(std::string suffix)}: saves the final image
\end{itemize}

\subsection{The \okt{ecn::Point} class}\label{ptClass}

This class implements basic properties of a 2D point:

\begin{itemize}
 \item \okt{bool operator==(Point)}: returns true if both points have the same x and y coordinates
 \item \okt{double h(const Point \&goal)}: heuristic distance, may use Manhattan
 ($|x-\okt{goal}.x| + |y-\okt{goal}.y|$) or classical Euclidean distance.
 \item \okt{void show(bool closed, const Point \& parent)}: draws a straight line between the point and its parent, that
 is blue if the point is in the closed set or red if it is in the open set.
 \item \okt{void print(const Point\& parent)}: writes the final path into the maze for display, also considers a straight line
\end{itemize}
All built classes should inherit from this class. The considered maze is available through the static member variable \okt{ecn::maze} and can thus
be accessed from the built member functions.

\end{document}
